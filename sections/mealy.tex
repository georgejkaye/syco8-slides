\section{Mealy machines}

\begin{frame}
    \frametitle{Mealy machine}

    Mealy machines are a kind of \alert{finite state machine}.

    \pause

    Sets of \alert{states} \(S\), \pause \alert{inputs} \(M\), \pause \alert{outputs} \(N\)

    \pause

    Given input \(m \in M\) and state \(s_1 \in S\) we have:
    
    \pause

    \begin{itemize}
        \item \alert{next state} \(T(s_1)(m) = s_2 \in S\)
        \pause
        \item \alert{output} \(O(s_1)(m) = n \in N\).
    \end{itemize}

    \pause

    A Mealy machine also has a \alert{start state}.

    \pause

    \begin{center}
        \includestandalone{tikz/streams/exmealy}
    \end{center}

\end{frame}

\begin{frame}
    \frametitle{Mealy machines: bisimilarity}

    Mealy machines have a notion of \alert{bisimilarity}.

    \pause

    If two machines are \alert{observationally equivalent}, then they are bisimilar.

\end{frame}

\begin{frame}
    \frametitle{Props of mealy machines}

    By setting \(M\) and \(N\) to powers of \(\values\), we can define a prop of Mealy machines.

    \pause

    A morphism \(m \to n\) is a Mealy machine with inputs \(\valuetuple{m}\) and outputs \(\valuetuple{n}\).
    
    \pause

    We can define composition, tensor, trace...

    \pause

    We interpret circuits as Mealy machines using a functor $\circuittomealy{\interpretation}$.

\end{frame}

\begin{frame}
    \frametitle{Interpreting \(\scirc[\Sigma]\)}    

    Source of \alert{state} in our circuits: \alert{values}...

    \pause

    \begin{center}
        \tikzfig{circuits/components/v}
        \qquad
        \pause
        $\xrightarrow{\circuittomealy{\interpretation}}$
        \qquad
        \raisebox{-1.25em}{\includestandalone{tikz/streams/valmealy}}
    \end{center}
\end{frame}

\begin{frame}
    \frametitle{Interpreting \(\scirc[\Sigma]\)}    
    ...and \alert{delay}.

    \pause
    \begin{center}
        \tikzfig{circuits/components/delay}
        \qquad
        \pause
        $\xrightarrow{\circuittomealy{\interpretation}}$
        \qquad
        \raisebox{-8em}{\includestandalone[scale=0.9]{tikz/streams/delaymealy}}
    \end{center}
\end{frame}


\begin{frame}
    \frametitle{Interpreting \(\scirc[\Sigma]\)}

    Gates don't have any `internal state'.

    \pause

    \begin{center}
        \tikzfig{circuits/components/gate}
        \qquad
        \pause
        $\xrightarrow{\circuittomealy{\interpretation}}$
        \qquad
        \raisebox{-1em}{\includestandalone{tikz/streams/gatemealy}}
    \end{center}

    \pause

    To interpret circuit morphisms, combine with composition, tensor, trace...

\end{frame}


\begin{frame}
    \frametitle{Interpreting \(\scirc[\Sigma]\)}

    Do all the axioms of \(\scirc[\Sigma]\) hold in this prop? \pause \alert{Yes.}

    \pause

    \begin{theorem}
        For any $F,G \in \scirc[\Sigma]$, if $F = G$ then $\circuittomealy{\interpretation}[F]$ and $\circuittomealy{\interpretation}[G]$ are bisimilar.
    \end{theorem}

    \pause

    Can we return to a circuit from an arbitrary Mealy machine? 

    \pause

    Should be possible, rudimentary task in circuit design.
 
    \pause

    But first, how do we get to \alert{streams}?

\end{frame}