\section{Categorical semantics for digital~circuits}

\begin{frame}
    \frametitle{Circuit signature}

    \pause

    \alert{Values} \(\values\) forming a lattice.

    \pause

    \[\values = \tikzfig{streams/lattice}\]

    \pause

    \alert{Gate symbols} \(g\) with associated monotonic functions \(\morph{\bar{g}}{\valuetuple{m}}{\values}\).

    \pause

    \[\{\morph{\ \mf{AND}}{\valuetuple{2}}{\values},\ \morph{\mf{OR}}{\valuetuple{2}}{\values}\ \}\]

\end{frame}

\begin{frame}
    \frametitle{Combinational circuits}

    \pause

    Circuits are \alert{morphisms} in the prop generated freely over a signature \(\Sigma\).

    \pause

    e.g:

    \begin{center}
        \tikzfig{circuits/components/bot}
        \tikzfig{circuits/components/t}
        \tikzfig{circuits/components/f}
        \tikzfig{circuits/components/top}

        \pause

        \vspace{1em}

        \tikzfig{circuits/components/and}
        \tikzfig{circuits/components/or}
    \end{center}

    \vspace{1em}

    \pause

    Along with \alert{structural} generators 

    \begin{center}
        \tikzfig{circuits/components/fork}
        \quad
        \tikzfig{circuits/components/join}
        \quad
        \tikzfig{circuits/components/stub}
    \end{center}

\end{frame}

\begin{frame}
    \frametitle{Combinational circuits: axioms}

    Behaviour defined using \alert{axioms}.

    \pause
    \begin{center}
        \vspace{2em}

        \scalebox{1}{\tikzfig{circuits/axioms/fork-lhs}}
        \quad\(=\)\quad
        \scalebox{1}{\tikzfig{circuits/axioms/fork-rhs}}
        %
        \hspace{2em}
        \pause
        %
        \scalebox{1}{\tikzfig{circuits/axioms/join-lhs}}
        \quad\(=\)\quad
        \scalebox{1}{\tikzfig{circuits/axioms/join-rhs}}
        
        \vspace{2em}
        \pause
        
        \scalebox{1}{\tikzfig{circuits/axioms/stub-lhs}}
        \quad\(=\)\quad
        \scalebox{1}{\tikzfig{circuits/axioms/stub-rhs}}
        %
        \pause
        \hspace{2em}
        %
        \scalebox{1}{\tikzfig{circuits/axioms/gate-lhs}}
        \quad\(=\)\quad
        \scalebox{1}{\tikzfig{circuits/axioms/gate-rhs}}

    \end{center}
\end{frame}

\begin{frame}
    \frametitle{Combinational circuits: extensional equivalence}

    We also consider \alert{input-output behaviour}.

    \pause

    \begin{center}
        \tikzfig{circuits/cartesian/extensional-equivalence-lhs}
        \quad\(=\)\quad
        \tikzfig{circuits/cartesian/extensional-equivalence-res}
        \quad\(=\)\quad
        \tikzfig{circuits/cartesian/extensional-equivalence-rhs}
    \end{center}
    
    \pause

    \(F\) and \(G\) are \alert{extensionally equivalent}.

\end{frame}

\begin{frame}
    \frametitle{The prop of combinational circuits}

    \begin{center}
        \scalebox{1.5}{\(\text{generators} \quad + \quad \text{axioms} \quad \)}
        
        \vspace{1em}
        
        \scalebox{1.5}{\(+ \quad \text{quotient by extensional equivalence} \quad = \quad \ccirc[\Sigma]\)}
    \end{center}
\end{frame}

\begin{frame}
    \frametitle{Temporal circuits}

    Combinational circuits are \alert{boring}.

    \pause

    \alert{Delay} is represented by a new generator 
    
    \vspace{2em}

    \begin{center}
        \scalebox{2}{\tikzfig{circuits/components/delay}}
    \end{center}

\end{frame}

\begin{frame}
    \frametitle{Temporal circuits: axioms}
    \begin{center}
        \vspace{2em}

        \scalebox{1}{\tikzfig{circuits/axioms/timelessness-lhs}}
        \quad\(=\)\quad
        \scalebox{1}{\tikzfig{circuits/axioms/timelessness-rhs}}
        %
        
        \vspace{2em}
        \pause

        \scalebox{1}{\tikzfig{circuits/axioms/unobservable-lhs}}
        \quad\(=\)\quad
        \scalebox{1}{\tikzfig{circuits/axioms/unobservable-rhs}}
        %
        \hspace{3.5em}
        \pause
        %
        \scalebox{1}{\tikzfig{circuits/axioms/disconnect-lhs}}
        \quad\(=\)\quad
        \scalebox{1}{\tikzfig{circuits/axioms/disconnect-rhs}}

        \vspace{2em}
        \pause

        %
        \scalebox{1}{\tikzfig{circuits/axioms/streaming-lhs}}
        \quad\(=\)\quad
        \scalebox{1}{\tikzfig{circuits/axioms/streaming-rhs}}

    \end{center}
\end{frame}

\begin{frame}
    \frametitle{The prop of temporal circuits}
    
    \begin{center}
        \scalebox{1.5}{\(\ccirc[\Sigma] + \scalebox{1}{\tikzfig{circuits/components/delay}} + \text{axioms} \quad = \quad \tcirc[\Sigma]\)}
    \end{center}

\end{frame}
\begin{frame}
    \frametitle{Sequential circuits}

    \pause

    Sequential circuits have delay and \alert{feedback}.
    
    \pause

    We freely add a \alert{trace operator}.

    \pause

    \begin{center}
        \tikzfig{strings/traced/trace-lhs}
        \quad\(\xrightarrow{\trace{1}{-}}\)\quad
        \tikzfig{strings/traced/trace-rhs}
    \end{center}        

    \pause

    \vspace{1.5em}

    \begin{center}
        \scalebox{1.5}{\(\tcirc[\Sigma] \quad + \quad \text{trace} \quad = \quad \scirc[\Sigma]\)}
    \end{center}

\end{frame}
\begin{frame}
    \frametitle{Sequential circuits}

    \pause

    \begin{theorem}[Ghica and Jung, 2016]
        \(\scirc[\Sigma]\) is cartesian.
    \end{theorem}
    \pause
    We can \alert{copy} and \alert{discard} data.
    \pause
    \begin{center}
        \tikzfig{strings/cartesian/naturality-copy-lhs}
        \pause
        \quad\(=\)\quad
        \tikzfig{strings/cartesian/naturality-copy-rhs}
        \quad\quad
        \pause
        \tikzfig{strings/cartesian/naturality-discard-lhs}
        \pause
        \quad\(=\)\quad
        \tikzfig{strings/cartesian/naturality-discard-rhs}
    \end{center}

\end{frame}

\begin{frame}
    \frametitle{Sequential circuits: unfolding}

    \pause

    Traced cartesian categories admit the \alert{unfolding} rule.

    \pause
    \vspace{1em}

    \begin{center}
        \tikzfig{strings/dataflow/unfolding-lhs}
        \pause
        \quad\(=\)\quad
        \tikzfig{strings/dataflow/unfolding-rhs}
    \end{center}

    \pause

    Crucial part of the operational semantics!
\end{frame}