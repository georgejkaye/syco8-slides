\documentclass[aspectratio=169]{beamer}

\usepackage{ stmaryrd }

\usepackage{ tikz }
\usepackage{ figures/tikzit }
\usepackage{ standalone }
\usetikzlibrary{decorations.markings}

\usetheme[
  background=light,
  numbering=counter,
  block=fill,
  %sectionpage=simple
]{metropolis}

\tikzset{
    invisible/.style={opacity=0},
    visible on/.style={alt={#1{}{invisible}}},
    alt/.code args={<#1>#2#3}{%
      \alt<#1>{\pgfkeysalso{#2}}{\pgfkeysalso{#3}}%
  }
}

\input{macros/letters}
\section{Categorical semantics for digital~circuits}

\begin{frame}
    \frametitle{Circuit signature}

    \pause

    \alert{Values} $\textbf{V}$ forming a lattice.

    \pause

    \alert{Gates} $g$ with associated monotonic functions $\morph{\bar{g}}{\textbf{V}^m}{\textbf{V}}$.


    \pause

    \[\textbf{V} = \tikzfig{streams/lattice}\]

    \pause

    \[\Sigma = (\textbf{V}, \{\morph{\ \mf{AND}}{\textbf{V}^2}{\textbf{V}},\ \morph{\mf{OR}}{\textbf{V}^2}{\textbf{V}}\ \})\]

\end{frame}

\begin{frame}
    \frametitle{Combinational circuits}

    \pause

    Circuits are morphisms generated freely over a signature $\Sigma$.

    \begin{center}
        \tikzfig{circuits/components/bot}
        \tikzfig{circuits/components/t}
        \tikzfig{circuits/components/f}
        \tikzfig{circuits/components/top}

        \pause

        \vspace{1em}

        \tikzfig{circuits/components/and}
        \tikzfig{circuits/components/or}
    \end{center}

    \vspace{1em}

    \pause

    Along with \alert{structural} generators 

    \begin{center}
        \tikzfig{circuits/components/fork}
        \quad
        \tikzfig{circuits/components/join}
        \quad
        \tikzfig{circuits/components/stub}
    \end{center}

    Composed together in sequence or parallel with \alert{identities} and \alert{symmetries}.

\end{frame}

\begin{frame}
    \frametitle{Combinational circuits: axioms}
    \begin{center}
        \vspace{2em}

        \scalebox{1}{\tikzfig{circuits/axioms/fork-lhs}}
        \quad$=$\quad
        \scalebox{1}{\tikzfig{circuits/axioms/fork-rhs}}
        %
        \hspace{2em}
        \pause
        %
        \scalebox{1}{\tikzfig{circuits/axioms/join-lhs}}
        \quad$=$\quad
        \scalebox{1}{\tikzfig{circuits/axioms/join-rhs}}
        
        \vspace{2em}
        \pause
        
        \scalebox{1}{\tikzfig{circuits/axioms/stub-lhs}}
        \quad$=$\quad
        \scalebox{1}{\tikzfig{circuits/axioms/stub-rhs}}
        %
        \pause
        \hspace{2em}
        %
        \scalebox{1}{\tikzfig{circuits/axioms/gate-lhs}}
        \quad$=$\quad
        \scalebox{1}{\tikzfig{circuits/axioms/gate-rhs}}

    \end{center}
\end{frame}

\begin{frame}
    \frametitle{Combinational circuits: extensional equivalence}

    \pause

    \begin{center}
        \tikzfig{circuits/cartesian/extensional-equivalence-lhs}
        \quad$=$\quad
        \tikzfig{circuits/cartesian/extensional-equivalence-res}
        \quad$=$\quad
        \tikzfig{circuits/cartesian/extensional-equivalence-rhs}
    \end{center}
    
    \pause

    $f$ and $f^\prime$ are \alert{extensionally equivalent}.

    \pause

    \vspace{1em}

    \begin{theorem}[Extensionality: Ghica and Jung, 2016]
        For any circuit \tikzfig{circuits/components/fcirc} and values \tikzfig{circuits/components/values}, there exists values \tikzfig{circuits/cartesian/extensional-equivalence-res} such that $\tikzfig{circuits/cartesian/extensional-equivalence-lhs} = \tikzfig{circuits/cartesian/extensional-equivalence-res}$.
    \end{theorem}
\end{frame}

\begin{frame}
    \frametitle{Combinational circuits}

    \begin{center}
        \scalebox{1.5}{$\text{generators} \quad + \quad \text{axioms} \quad $}
        
        \vspace{1em}
        
        \scalebox{1.5}{$+ \quad \text{extensional equivalence} \quad = \quad \ccirc[\Sigma]$}
    \end{center}
\end{frame}

\begin{frame}
    \frametitle{Temporal circuits}

    Combinational circuits are \alert{boring}.

    \pause

    \alert{Delay} is represented by a new generator 
    
    \vspace{2em}

    \begin{center}
        \scalebox{2}{\tikzfig{circuits/components/delay}}
    \end{center}

\end{frame}

\begin{frame}
    \frametitle{Temporal circuits: axioms}
    \begin{center}
        \vspace{2em}

        \scalebox{1}{\tikzfig{circuits/axioms/timelessness-lhs}}
        \quad$=$\quad
        \scalebox{1}{\tikzfig{circuits/axioms/timelessness-rhs}}
        %
        
        \vspace{2em}
        \pause

        \scalebox{1}{\tikzfig{circuits/axioms/unobservable-lhs}}
        \quad$=$\quad
        \scalebox{1}{\tikzfig{circuits/axioms/unobservable-rhs}}
        %
        \hspace{3.5em}
        \pause
        %
        \scalebox{1}{\tikzfig{circuits/axioms/disconnect-lhs}}
        \quad$=$\quad
        \scalebox{1}{\tikzfig{circuits/axioms/disconnect-rhs}}

        \vspace{2em}
        \pause

        %
        \scalebox{1}{\tikzfig{circuits/axioms/streaming-lhs}}
        \quad$=$\quad
        \scalebox{1}{\tikzfig{circuits/axioms/streaming-rhs}}

    \end{center}
\end{frame}

\begin{frame}
    \frametitle{Temporal circuits}
    
    \begin{center}
        \scalebox{1.5}{$\ccirc[\Sigma] + \scalebox{1}{\tikzfig{circuits/components/delay}} + \text{axioms} \quad = \quad \tcirc[\Sigma]$}
    \end{center}

\end{frame}
\begin{frame}
    \frametitle{Sequential circuits}

    \pause

    Sequential circuits have delay and \alert{feedback}.
    
    \pause

    We freely add a \alert{trace operator}.

    \pause

    \begin{center}
        \tikzfig{strings/traced/trace-lhs}
        \quad$\xrightarrow{\trace{1}{-}}$\quad
        \tikzfig{strings/traced/trace-rhs}
    \end{center}        

    \pause

    \vspace{1.5em}

    \begin{center}
        \scalebox{1.5}{$\tcirc[\Sigma] \quad + \quad \text{trace} \quad = \quad \scirc[\Sigma]$}
    \end{center}

\end{frame}
\begin{frame}
    \frametitle{Sequential circuits}

    \pause

    \begin{theorem}[Ghica and Jung, 2016]
        $\scirc[\Sigma]$ is cartesian.
    \end{theorem}
    \pause
    \begin{center}
        \tikzfig{strings/cartesian/naturality-copy-lhs}
        \pause
        \quad$=$\quad
        \tikzfig{strings/cartesian/naturality-copy-rhs}
        \quad\quad
        \pause
        \tikzfig{strings/cartesian/naturality-discard-lhs}
        \pause
        \quad$=$\quad
        \tikzfig{strings/cartesian/naturality-discard-rhs}
    \end{center}
    \pause
    We can \alert{copy} and \alert{discard} data.

\end{frame}

\begin{frame}
    \frametitle{Sequential circuits: unfolding}

    \pause

    Traced cartesian categories admit the \alert{unfolding} rule.

    \pause
    \vspace{1em}

    \begin{center}
        \tikzfig{strings/dataflow/unfolding-lhs}
        \pause
        \quad$=$\quad
        \tikzfig{strings/dataflow/unfolding-rhs}
    \end{center}

    \pause

    Crucial part of the operational semantics!
\end{frame}
\input{macros/category}
\section{Categorical semantics for digital~circuits}

\begin{frame}
    \frametitle{Circuit signature}

    \pause

    \alert{Values} $\textbf{V}$ forming a lattice.

    \pause

    \alert{Gates} $g$ with associated monotonic functions $\morph{\bar{g}}{\textbf{V}^m}{\textbf{V}}$.


    \pause

    \[\textbf{V} = \tikzfig{streams/lattice}\]

    \pause

    \[\Sigma = (\textbf{V}, \{\morph{\ \mf{AND}}{\textbf{V}^2}{\textbf{V}},\ \morph{\mf{OR}}{\textbf{V}^2}{\textbf{V}}\ \})\]

\end{frame}

\begin{frame}
    \frametitle{Combinational circuits}

    \pause

    Circuits are morphisms generated freely over a signature $\Sigma$.

    \begin{center}
        \tikzfig{circuits/components/bot}
        \tikzfig{circuits/components/t}
        \tikzfig{circuits/components/f}
        \tikzfig{circuits/components/top}

        \pause

        \vspace{1em}

        \tikzfig{circuits/components/and}
        \tikzfig{circuits/components/or}
    \end{center}

    \vspace{1em}

    \pause

    Along with \alert{structural} generators 

    \begin{center}
        \tikzfig{circuits/components/fork}
        \quad
        \tikzfig{circuits/components/join}
        \quad
        \tikzfig{circuits/components/stub}
    \end{center}

    Composed together in sequence or parallel with \alert{identities} and \alert{symmetries}.

\end{frame}

\begin{frame}
    \frametitle{Combinational circuits: axioms}
    \begin{center}
        \vspace{2em}

        \scalebox{1}{\tikzfig{circuits/axioms/fork-lhs}}
        \quad$=$\quad
        \scalebox{1}{\tikzfig{circuits/axioms/fork-rhs}}
        %
        \hspace{2em}
        \pause
        %
        \scalebox{1}{\tikzfig{circuits/axioms/join-lhs}}
        \quad$=$\quad
        \scalebox{1}{\tikzfig{circuits/axioms/join-rhs}}
        
        \vspace{2em}
        \pause
        
        \scalebox{1}{\tikzfig{circuits/axioms/stub-lhs}}
        \quad$=$\quad
        \scalebox{1}{\tikzfig{circuits/axioms/stub-rhs}}
        %
        \pause
        \hspace{2em}
        %
        \scalebox{1}{\tikzfig{circuits/axioms/gate-lhs}}
        \quad$=$\quad
        \scalebox{1}{\tikzfig{circuits/axioms/gate-rhs}}

    \end{center}
\end{frame}

\begin{frame}
    \frametitle{Combinational circuits: extensional equivalence}

    \pause

    \begin{center}
        \tikzfig{circuits/cartesian/extensional-equivalence-lhs}
        \quad$=$\quad
        \tikzfig{circuits/cartesian/extensional-equivalence-res}
        \quad$=$\quad
        \tikzfig{circuits/cartesian/extensional-equivalence-rhs}
    \end{center}
    
    \pause

    $f$ and $f^\prime$ are \alert{extensionally equivalent}.

    \pause

    \vspace{1em}

    \begin{theorem}[Extensionality: Ghica and Jung, 2016]
        For any circuit \tikzfig{circuits/components/fcirc} and values \tikzfig{circuits/components/values}, there exists values \tikzfig{circuits/cartesian/extensional-equivalence-res} such that $\tikzfig{circuits/cartesian/extensional-equivalence-lhs} = \tikzfig{circuits/cartesian/extensional-equivalence-res}$.
    \end{theorem}
\end{frame}

\begin{frame}
    \frametitle{Combinational circuits}

    \begin{center}
        \scalebox{1.5}{$\text{generators} \quad + \quad \text{axioms} \quad $}
        
        \vspace{1em}
        
        \scalebox{1.5}{$+ \quad \text{extensional equivalence} \quad = \quad \ccirc[\Sigma]$}
    \end{center}
\end{frame}

\begin{frame}
    \frametitle{Temporal circuits}

    Combinational circuits are \alert{boring}.

    \pause

    \alert{Delay} is represented by a new generator 
    
    \vspace{2em}

    \begin{center}
        \scalebox{2}{\tikzfig{circuits/components/delay}}
    \end{center}

\end{frame}

\begin{frame}
    \frametitle{Temporal circuits: axioms}
    \begin{center}
        \vspace{2em}

        \scalebox{1}{\tikzfig{circuits/axioms/timelessness-lhs}}
        \quad$=$\quad
        \scalebox{1}{\tikzfig{circuits/axioms/timelessness-rhs}}
        %
        
        \vspace{2em}
        \pause

        \scalebox{1}{\tikzfig{circuits/axioms/unobservable-lhs}}
        \quad$=$\quad
        \scalebox{1}{\tikzfig{circuits/axioms/unobservable-rhs}}
        %
        \hspace{3.5em}
        \pause
        %
        \scalebox{1}{\tikzfig{circuits/axioms/disconnect-lhs}}
        \quad$=$\quad
        \scalebox{1}{\tikzfig{circuits/axioms/disconnect-rhs}}

        \vspace{2em}
        \pause

        %
        \scalebox{1}{\tikzfig{circuits/axioms/streaming-lhs}}
        \quad$=$\quad
        \scalebox{1}{\tikzfig{circuits/axioms/streaming-rhs}}

    \end{center}
\end{frame}

\begin{frame}
    \frametitle{Temporal circuits}
    
    \begin{center}
        \scalebox{1.5}{$\ccirc[\Sigma] + \scalebox{1}{\tikzfig{circuits/components/delay}} + \text{axioms} \quad = \quad \tcirc[\Sigma]$}
    \end{center}

\end{frame}
\begin{frame}
    \frametitle{Sequential circuits}

    \pause

    Sequential circuits have delay and \alert{feedback}.
    
    \pause

    We freely add a \alert{trace operator}.

    \pause

    \begin{center}
        \tikzfig{strings/traced/trace-lhs}
        \quad$\xrightarrow{\trace{1}{-}}$\quad
        \tikzfig{strings/traced/trace-rhs}
    \end{center}        

    \pause

    \vspace{1.5em}

    \begin{center}
        \scalebox{1.5}{$\tcirc[\Sigma] \quad + \quad \text{trace} \quad = \quad \scirc[\Sigma]$}
    \end{center}

\end{frame}
\begin{frame}
    \frametitle{Sequential circuits}

    \pause

    \begin{theorem}[Ghica and Jung, 2016]
        $\scirc[\Sigma]$ is cartesian.
    \end{theorem}
    \pause
    \begin{center}
        \tikzfig{strings/cartesian/naturality-copy-lhs}
        \pause
        \quad$=$\quad
        \tikzfig{strings/cartesian/naturality-copy-rhs}
        \quad\quad
        \pause
        \tikzfig{strings/cartesian/naturality-discard-lhs}
        \pause
        \quad$=$\quad
        \tikzfig{strings/cartesian/naturality-discard-rhs}
    \end{center}
    \pause
    We can \alert{copy} and \alert{discard} data.

\end{frame}

\begin{frame}
    \frametitle{Sequential circuits: unfolding}

    \pause

    Traced cartesian categories admit the \alert{unfolding} rule.

    \pause
    \vspace{1em}

    \begin{center}
        \tikzfig{strings/dataflow/unfolding-lhs}
        \pause
        \quad$=$\quad
        \tikzfig{strings/dataflow/unfolding-rhs}
    \end{center}

    \pause

    Crucial part of the operational semantics!
\end{frame}

% FiraFonts
\usepackage[sfdefault]{FiraSans}
\usepackage{FiraMono}
% Use thinner fonts
\makeatletter
\def\bfseries@sf{medium}
\def\mdseries@sf{l}
\makeatother

\definecolor{backg}{RGB}{9,72,61}
\definecolor{accent}{RGB}{0,150,136}

\definecolor{dracback}{RGB}{40, 42, 54}
\definecolor{dracfore}{RGB}{248, 248, 242}
\definecolor{dractitle}{RGB}{56, 58, 89}
\definecolor{dracblock}{RGB}{98, 114, 164}
\definecolor{draccent}{RGB}{255, 121, 198}

\setbeamercolor{normal text}{bg=dracfore}
\setbeamercolor{frametitle}{bg=dractitle, fg=dracfore}
\setbeamercolor{title separator}{fg=draccent}
\setbeamercolor{progress bar}{fg=draccent, bg=draccent}
\setbeamercolor{block title}{fg=dracfore, bg=dracblock}
\setbeamercolor{alerted text}{fg=draccent}

\newtheorem{proposition}{Proposition}

\title{Normalisation by evaluation for digital circuits}
\author{\texorpdfstring{\large\textbf{George Kaye}, Dan R. Ghica, David Sprunger \\ \normalsize University of Birmingham}{George Kaye}}
\institute{SYCO 8}
\date{13 December 2021}

\begin{document}

    \maketitle

    \begin{frame}
        \frametitle{Hello!}
    
        
    
    \end{frame}

    \section{Categorical semantics for digital~circuits}

    \begin{frame}
        \frametitle{Circuit signature}
    
        \pause

        \alert{Values} $\textbf{V}$ forming a lattice.

        \pause

        \alert{Gates} $g$ with associated monotonic functions $\morph{\bar{g}}{\textbf{V}^m}{\textbf{V}}$.


        \pause

        \[\textbf{V} = \tikzfig{streams/lattice}\]

        \pause

        \[\Sigma = (\textbf{V}, \{\morph{\ \mf{AND}}{\textbf{V}^2}{\textbf{V}},\ \morph{\mf{OR}}{\textbf{V}^2}{\textbf{V}}\ \})\]

    \end{frame}

    \begin{frame}
        \frametitle{Combinational circuits}

        \pause

        Circuits are morphisms generated freely over a signature $\Sigma$.

        \begin{center}
            \tikzfig{circuits/components/bot}
            \tikzfig{circuits/components/t}
            \tikzfig{circuits/components/f}
            \tikzfig{circuits/components/top}

            \pause

            \vspace{1em}

            \tikzfig{circuits/components/and}
            \tikzfig{circuits/components/or}
        \end{center}

        \vspace{1em}

        \pause

        Alonmg with \alert{structural} generators 

        \begin{center}
            \tikzfig{circuits/components/fork}
            \quad
            \tikzfig{circuits/components/join}
            \quad
            \tikzfig{circuits/components/stub}
        \end{center}

        Composed together in sequence or parallel with \alert{identities} and \alert{symmetries}.

    \end{frame}

    \begin{frame}
        \frametitle{Combinational circuits: axioms}
        \begin{center}
            \vspace{2em}

            \scalebox{1}{\tikzfig{circuits/axioms/fork-lhs}}
            \quad$=$\quad
            \scalebox{1}{\tikzfig{circuits/axioms/fork-rhs}}
            %
            \hspace{2em}
            \pause
            %
            \scalebox{1}{\tikzfig{circuits/axioms/join-lhs}}
            \quad$=$\quad
            \scalebox{1}{\tikzfig{circuits/axioms/join-rhs}}
            
            \vspace{2em}
            \pause
            
            \scalebox{1}{\tikzfig{circuits/axioms/stub-lhs}}
            \quad$=$\quad
            \scalebox{1}{\tikzfig{circuits/axioms/stub-rhs}}
            %
            \hspace{2em}
            %
            \scalebox{1}{\tikzfig{circuits/axioms/gate-lhs}}
            \quad$=$\quad
            \scalebox{1}{\tikzfig{circuits/axioms/gate-rhs}}

        \end{center}
    \end{frame}

    \begin{frame}
        \frametitle{Combinational circuits: extensional equivalence}

        \pause

        \begin{center}
            \tikzfig{circuits/cartesian/extensional-equivalence-lhs}
            \quad$=$\quad
            \tikzfig{circuits/cartesian/extensional-equivalence-res}
            \quad$=$\quad
            \tikzfig{circuits/cartesian/extensional-equivalence-rhs}
        \end{center}
        
        \pause

        $f$ and $f^\prime$ are \alert{extensionally equivalent}.

        \pause

        \vspace{1em}

        \begin{theorem}[Extensionality: Ghica and Jung, 2016]
            For any circuit \tikzfig{circuits/components/circuit} and values \tikzfig{circuits/components/values}, there exists values \tikzfig{circuits/cartesian/extensional-equivalence-res} such that \tikzfig{circuits/cartesian/extensional-equivalence-lhs} = \tikzfig{circuits/cartesian/extensional-equivalence-res}.
        \end{theorem}
    \end{frame}

    \begin{frame}
        \frametitle{Combinational circuits}
    
        \begin{center}
            \scalebox{1.5}{$\text{generators} \quad + \quad \text{axioms} \quad $}
            
            \vspace{1em}
            
            \scalebox{1.5}{$+ \quad \text{extensional equivalence} \quad = \quad \ccirc[\Sigma]$}
        \end{center}
    \end{frame}

    \begin{frame}
        \frametitle{Temporal circuits}
    
        Combinational circuits are \alert{boring}.

        \alert{Delay} is represented by a new generator 
        
        \vspace{2em}

        \begin{center}
            \scalebox{2}{\tikzfig{circuits/components/delay}}
        \end{center}

    \end{frame}

    \begin{frame}
        \frametitle{Temporal circuits: axioms}
        \begin{center}
            \vspace{2em}

            \scalebox{1}{\tikzfig{circuits/axioms/timelessness-lhs}}
            \quad$=$\quad
            \scalebox{1}{\tikzfig{circuits/axioms/timelessness-rhs}}
            %
            
            \vspace{2em}
            \pause

            \scalebox{1}{\tikzfig{circuits/axioms/unobservable-lhs}}
            \quad$=$\quad
            \scalebox{1}{\tikzfig{circuits/axioms/unobservable-rhs}}
            %
            \hspace{3.5em}
            \pause
            %
            \scalebox{1}{\tikzfig{circuits/axioms/disconnect-lhs}}
            \quad$=$\quad
            \scalebox{1}{\tikzfig{circuits/axioms/disconnect-rhs}}

            \vspace{2em}
            \pause

            %
            \scalebox{1}{\tikzfig{circuits/axioms/streaming-lhs}}
            \quad$=$\quad
            \scalebox{1}{\tikzfig{circuits/axioms/streaming-rhs}}

        \end{center}
    \end{frame}

    \begin{frame}
        \frametitle{Temporal circuits}
     
        \begin{center}
            \scalebox{1.5}{$\ccirc[\Sigma] \quad + \scalebox{1}{\tikzfig{circuits/components/delay}} + \quad \text{axioms} \quad = \quad \tcirc[\Sigma]$}
        \end{center}
    
    \end{frame}
    \begin{frame}
        \frametitle{Sequential circuits}
    
        \pause

        Sequential circuits have delay and \alert{feedback}.
        
        \pause

        We freely add a \alert{trace operator}.

        \pause

        \begin{center}
            \tikzfig{strings/traced/trace-lhs}
            \quad$\xrightarrow{\trace{1}{-}}$\quad
            \tikzfig{strings/traced/trace-rhs}
        \end{center}        

        \pause

        \vspace{1.5em}

        \begin{center}
            \scalebox{1.5}{$\tcirc[\Sigma] \quad + \quad \text{trace} \quad = \quad \scirc[\Sigma]$}
        \end{center}

    \end{frame}
    \begin{frame}
        \frametitle{Sequential circuits}

        \pause

        \begin{theorem}[Ghica and Jung, 2016]
            $\scirc[\Sigma]$ is cartesian.
        \end{theorem}
        \pause
        \begin{center}
            \tikzfig{strings/cartesian/naturality-copy-lhs}
            \quad$=$\quad
            \tikzfig{strings/cartesian/naturality-copy-rhs}
            \quad\quad
            \tikzfig{strings/cartesian/naturality-discard-lhs}
            \quad$=$\quad
            \tikzfig{strings/cartesian/naturality-discard-rhs}
        \end{center}
        \pause
        We can \alert{copy} and \alert{discard} data.

    \end{frame}

    \begin{frame}
        \frametitle{Sequential circuits: unfolding}

        \pause

        Traced cartesian categories admit the \alert{unfolding} rule.

        \pause
        \vspace{1em}

        \begin{center}
            \tikzfig{strings/dataflow/unfolding-lhs}
            \quad$=$\quad
            \tikzfig{strings/dataflow/unfolding-rhs}
        \end{center}

        \pause

        Crucial part of the operational semantics!
    \end{frame}

    \section{Evaluating digital circuits}

    \begin{frame}
        \frametitle{Productivity}
    
        

        This isn't nece
    
    \end{frame}

    \begin{frame}
        \frametitle{Delay-guarded feedback}
    
        
    
    \end{frame}

    \begin{frame}
        \frametitle{`Instant' feedback}
    
        
    
    \end{frame}

    \begin{frame}
        \frametitle{Unproductive circuits}
    
        
    
    \end{frame}

    \begin{frame}
        \frametitle{Eliminating `instant' feedback}
    
        
    
    \end{frame}


    \section{Mealy machines}


    \section{Streams}


    \section{Open circuits}

    \begin{frame}
        \frametitle{Open circuits}
    
        The next step...
    
    \end{frame}

\end{document}